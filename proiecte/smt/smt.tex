% vim: set tw=78 aw:
\documentclass{beamer}

\usepackage[utf8x]{inputenc} % diacritice
\usepackage[romanian]{babel}
\usepackage{color}			 % highlight
\usepackage{alltt}			 % highlight
\usepackage{code/highlight}	 % highlight
\usepackage{hyperref}        % folositi \url{http://...}
                             % sau \href{http://...}{Nume Link}
\mode<presentation>
{ \usetheme{Rochester} }		% TODO: settle this

% Titlul nu foloseşte Unicode pentru că e o problemă căreia nu i-am dat de
% cap.
\title[SMT]{Simple MP3 Trimmer}
\subtitle{CDL - Proiecte}
\institute{ROSEdu}
\author{Alex Eftimie \texttt{alex@eftimie.ro}}

\begin{document}

% Slide-urile cu mai multe părţi sunt marcate cu textul (cont.)
\setbeamertemplate{frametitle continuation}[from second]
% Arătăm numărul frame-ului
\setbeamertemplate{footline}[frame number]

\frame{\titlepage}

%\frame{\tableofcontents}

% NB: Secţiunile nu sunt marcate vizual, ci doar apar în cuprins.
%\section{Distribuit vs. centralizat}

% Pentru reamintirea periodică a cuprinsului şi unde ne aflăm:
%\frame{\tableofcontents[currentsection]}

% Titlul unui frame se specifică fie în acolade, imediat după \begin{frame},
% fie folosind \frametitle
\begin{frame}{Simple MP3 Trimmer}
\begin{itemize} % Just like normal LaTeX
\item O aplicație grafică simplă cu un scop precis: decuparea fișierelor audio.
\item Limbaj: Python
\item Tehnologii: GTK
\end{itemize}
\end{frame}

\end{document}
