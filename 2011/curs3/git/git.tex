% vim: set tw=78 aw sw=2 sts=2 noet:
\documentclass{beamer}

%\includeonlyframes{c} -- speeding up compilation speed during debug

\usepackage[utf8x]{inputenc} % diacritice
\usepackage[romanian]{babel}
\usepackage{hyperref}        % folositi \url{http://...}
\mode<presentation>
\usetheme{CDL}

\title[]{Git}
\subtitle{CDL 2011 - Cursul 3}
\institute[]{ROSEdu}
\author[]{Mihai Maruseac \\ \texttt{mihai@rosedu.org}}

\setbeamertemplate{frametitle continuation}[from second]
\setbeamertemplate{footline}[frame number]

%\pgfdeclareimage[height=5cm]{m0}{img/m0}
%\pgfdeclareimage[height=7cm]{m1}{img/m1}
%\pgfdeclareimage[height=7cm]{m2}{img/m2}
%\pgfdeclareimage[height=7cm]{m3}{img/m3}

\begin{document}

\maketitle

\section{Source Control Management}

\begin{frame}{Source Control Management}
  \begin{itemize}[<+->]
    \item aka Revision Control
    \item urmărire/stocare modificări
      \begin{alertblock}{Careful!}
        NU copy-paste în directoare separate
      \end{alertblock}
    \item colaborare, proiecte mari
      \begin{alertblock}{Careful!}
        NU mail / dropbox
      \end{alertblock}
  \end{itemize}
\end{frame}

\begin{frame}{Concepte}
  \begin{description}[<+->]
    \item[repository] = ...
    \item[checkout,clone] = make local copy
    \item[working copy] = copie locală, \textbf{sandbox}
    \item[change] = modificare
    \item[commit] = submit changes
    \item[update] = sync
    \item[merge] = aplicare a 2+ commits
    \item[conflict] = schimbări în același document
    \item[version] = revision
    \item[HEAD] = cel mai recent commit (tip)
    \item[branch] = alternativă development
    \item[label] = punct important din development (tag)
  \end{description}
\end{frame}

\begin{frame}{Unde e repository-ul?}
  \begin{itemize}
    \item centralizat
      \begin{itemize}
        \item server: DB
        \item client: working copy
      \end{itemize}
      \pause
      \begin{description}
        \item[+] totul la un loc
        \item[--] bottleneck, \textit{single point of failure}
      \end{description}
    \pause
    \item descentralizat
      \begin{itemize}
        \item P2P
        \item operații/concepte noi
        \item repo si working-tree simultan
        \item istoric si lucru complet descentralizat
      \end{itemize}
  \end{itemize}
\end{frame}

\begin{frame}{SCM Descentralizat}
  \begin{itemize}
    \item avantaje
      \begin{itemize}
        \item fără \textit{single point of failure}
        \item lucru deconectat
        \item operații rapide
      \end{itemize}
    \pause
    \item dezavantaje
      \begin{itemize}
        \item mai greu de înteles
        \item cine are acces?
        \item istoria e permanentă (almost)
      \end{itemize}
    \pause
    \item operatii noi
      \begin{description}[<+->]
        \item[push] - sync eu $\rightarrow$ remote
        \item[fetch,pull] - sync eu $\leftarrow$ remote
        \item[remote] - ...
      \end{description}
  \end{itemize}
\end{frame}

\begin{frame}{SCMs}
  \begin{itemize}
    \item \textbf{SVN} (Subversion), centralizat, C
    \item \textbf{Git} descentralizat, Perl, C, shell
    \item \textbf{Mercurial} descentralizat, $<$ Git (see wiki ref), Python, C, shell
    \item \textbf{darcs} patch (see Darcs ref), Haskell, lazy
  \end{itemize}
  \pause
  \begin{alertblock}{Hint}
    Toate sunt folosite. Încercați-le și voi :)
  \end{alertblock}
\end{frame}

\section{Git}

\begin{frame}{Git history}
  \begin{itemize}
    \item Linus Torvalds
    \item Linux Kernel (moved from BitKeeper)
    \item design:
      \begin{itemize}
        \item \textit{CVS as example of what \textbf{not} to do}
        \item distribuit, workflow a la BitKeeper
        \item protecție contra distrugerilor/erorilor
        \item perfomanță
      \end{itemize}
    \item development start: 3 aprilie 2005
    \item 29 aprilie 2005: 6.7 patch-uri/s în kernel via Git
    \item 16 iunie 2005: 2.6.12 pe Git
    \item 31 ianuarie 2011: git 1.7.4
    \item C, Perl, shell
  \end{itemize}
\end{frame}

\begin{frame}{Git mits}
  \begin{itemize}[<+->]
    \item git is hard to learn
    \item git is slow
    \item git nu e matur
    \item nimeni nu folosește git
    \item proiectele care folosesc git au un SVN in spate
    \item nu poti muta proiecte de pe SVN pe Git
  \end{itemize}
\end{frame}

\begin{frame}{Git metadata}
  \begin{itemize}[<+->]
    \item director \texttt{.git} \textit{doar} în rădăcină
    \item conținut:
      \begin{itemize}
        \item \texttt{.git/config} fișier configurare
        \item \texttt{.git/hooks} scripturi pentru evenimente
        \item \texttt{.git/refs} referințe (la ce?)
      \end{itemize}
  \end{itemize}
\end{frame}

\begin{frame}{Git config}
  \begin{itemize}
    \item local în \texttt{.git/config}
    \item global în \texttt{\$HOME/.gitconfig}
  \end{itemize}
  \pause
  \begin{alertblock}{Task 1}
    \begin{itemize}
      \item install git
      \item \texttt{git config --global user.name "\textit{nume prenume}"}
      \item \texttt{git config --global user.email "\textit{nume@dom.com}"}
      \item \texttt{git config --global color.ui auto}
      \item \texttt{git config --global color.pager true}
      \item \texttt{git config --global core.editor \textit{editor}}
    \end{itemize}
  \end{alertblock}
\end{frame}

\begin{frame}{Git commands}
  \begin{itemize}
    \item \texttt{git \textit{cmd args}}
    \item \texttt{git help}
    \item \texttt{git help \textit{cmd}}
    \item \texttt{git \textit{cmd} help}
    \item \texttt{man git-\texttt{cmd}}
  \end{itemize}
  \pause
  \begin{alertblock}{Task 2}
    \begin{itemize}
      \item Ce reprezenta opțiunea \texttt{color.pager} de la \texttt{git config}?
    \end{itemize}
  \end{alertblock}
\end{frame}

\begin{frame}{Primul repo}
  \begin{itemize}
    \item \texttt{git init} într-un director cu surse
    \item \texttt{git clone \textit{url}}
  \end{itemize}
  \pause
  \begin{alertblock}{Task 3}
    \begin{itemize}
      \item Obțineți repository-ul de la adresa TODO
      \item listați conținutul directorului creat
    \end{itemize}
  \end{alertblock}
\end{frame}

\begin{frame}{Conținut}
  \begin{alertblock}{Careful!}
    Git urmărește conținut, nu fișiere!!
  \end{alertblock}
  \pause
  \begin{itemize}
    \item work tree (working dir): local
    \item index: pregătit de commit
    \item repository: commit salvat
  \end{itemize}
\end{frame}

\begin{frame}{Staging}
  \begin{itemize}
    \item work tree $\rightarrow$ index
    \item \texttt{git add \textit{filename}}
    \item \texttt{git rm \textit{filename}}
    \item \texttt{git mv \textit{filename}}
  \end{itemize}
  \pause
  \begin{alertblock}{Task 4}
    \begin{itemize}
      \item Modificați sursa TODO din director, adăugând numele vostru pe linia
      corespunzătoare.
      \item Mutați în index fișierul modificat
    \end{itemize}
  \end{alertblock}
\end{frame}

\begin{frame}{Commiting}
  \begin{itemize}
    \item index $\rightarrow$ repository \textbf{local}
    \item \texttt{git commit \textbf{-m "\textit{mesaj}"}}
  \end{itemize}
  \begin{alertblock}{Careful!}
    Nu folosiți mesaje goale!
  \end{alertblock}
\end{frame}

\begin{frame}{Shortcuts}
  \begin{itemize}
    \item \texttt{git commit -am "..."}
    \item \texttt{git add .}
    \pause
    \begin{alertblock}{Tip}
      Nu rulați \texttt{git commit -am..}.
    \end{alertblock}
    \begin{alertblock}{Careful!}
      Nu rulați \texttt{git add .} dacă nu aveți totul setat ok.
    \end{alertblock}
    \pause
    \item probleme:
      \begin{itemize}
        \item fișiere executabile (\texttt{a.out}), obiect (\texttt{a.o})
        \item fișiere swap (\texttt{.git.tex.swp})
        \item fișiere locale, personale
        \item etc.
      \end{itemize}
  \end{itemize}
\end{frame}

\begin{frame}{Ignore}
  \begin{itemize}
    \item \texttt{.gitignore} - \textbf{global!!}
    \item \texttt{.git/info/exclude} - local
  \end{itemize}
  \pause
  \begin{alertblock}{Task 5}
    \begin{itemize}
      \item Ignorați \textbf{local} fișierele obiect și executabilul generat.
    \end{itemize}
  \end{alertblock}
\end{frame}

\begin{frame}{Commit (2)}
  \begin{itemize}
    \item fiecare commit are un ID: SHA1
    \item fiecare commit are cel puțin un părinte (cu excepția primului)
    \item graf aciclic de commit-uri
    \pause
    \item undo: \texttt{git reset}
    \pause
    \item cum referim un commit?
  \end{itemize}
\end{frame}

\begin{frame}{Commit (3)}
  \begin{description}
    \item[\texttt{HEAD}] = ultimul commit
    \item[\texttt{HEAD\textasciicircum}] = penultimul
    \item[\texttt{HEAD\textasciicircum\textasciicircum}] = antepenultimul
    \item[...] = etc
    \pause
    \item[\texttt{HEAD\textasciitilde5}] = \texttt{HEAD\textasciicircum\textasciicircum\textasciicircum\textasciicircum\textasciicircum}
    \pause
    \item[a946e644d5b4c26aaa4e73338805e207bbfd78b0] = full hash
    \pause
    \item[a946e6] = short hash
  \end{description}
\end{frame}

\begin{frame}{Status}
  \begin{itemize}
    \item \texttt{git status}
    \item prezintă:
      \begin{itemize}
        \item conținut din index {staged}
        \item conținut care nu e în index {non-staged}
        \item conținut neurmărit (non-tracked)
        \item other info
      \end{itemize}
  \end{itemize}
  \pause
  \begin{alertblock}{Task 6}
    \begin{itemize}
      \item Modificați sursa TODO, adăugând numele și grupa pe linia
      corespunzătoare.
      \item creați un fișier numit după voi în working dir
      \item vizualizați starea working dir.
      \item faceți astfel încât
    \end{itemize}
  \end{alertblock}
\end{frame}

\section{More git}

\begin{frame}{Bibliografie}
  \begin{itemize}
    \item \href{http://en.wikipedia.org/wiki/Comparison_of_revision_control_software}{Comparație între SCM-uri}
    \item \href{http://en.wikibooks.org/wiki/Understanding_Darcs/Patch_theory}{Darcs Patch Theory}
  \end{itemize}
\end{frame}

\section{Extra git}
\section{Even more extra git}

%\begin{frame}{Before we begin}
%  \begin{block}{$>$ echo "Test" $|$ grep T}
%    Test\\
%    (linux.c, 36): Cannot wait for child: No child processes
%  \end{block}
%  \begin{block}{$>$ echo "Test" $|$ grep T}
%    Test\\
%    $>$ exit
%  \end{block}
%\end{frame}

%\begin{frame}{Studiu de caz}
%  \transwipe[direction=90,duration=2]
%  \begin{pgfpicture}{0cm}{0cm}{10cm}{5cm}
%    \pgfsetcolor{red}
%    \pgfrect[stroke]{\pgfpoint{3cm}{1cm}}{\pgfpoint{6cm}{1.5cm}}
%  \end{pgfpicture}
%
%  1. Pornim cu un nucleu mic...
%\end{frame}

%\begin{frame}{Studiu de caz}
%  \begin{pgfpicture}{0cm}{0cm}{10cm}{5cm}
%    \pgfsetcolor{red}
%    \pgfline{\pgfpoint{3cm}{1cm}}{\pgfpoint{3cm}{2.5cm}}
%    \pgfline{\pgfpoint{3cm}{2.5cm}}{\pgfpoint{5cm}{2.5cm}}
%    \pgfline{\pgfpoint{5cm}{2.5cm}}{\pgfpoint{6cm}{1.5cm}}
%    \pgfline{\pgfpoint{6cm}{1.5cm}}{\pgfpoint{7cm}{2.5cm}}
%    \pgfline{\pgfpoint{7cm}{2.5cm}}{\pgfpoint{9cm}{2.5cm}}
%    \pgfline{\pgfpoint{9cm}{2.5cm}}{\pgfpoint{9cm}{1cm}}
%    \pgfline{\pgfpoint{9cm}{1cm}}{\pgfpoint{3cm}{1cm}}
%    \pgfsetcolor{blue}
%    \pgfline{\pgfpoint{5cm}{2.52cm}}{\pgfpoint{5cm}{4cm}}
%    \pgfline{\pgfpoint{5cm}{4cm}}{\pgfpoint{7cm}{4cm}}
%    \pgfline{\pgfpoint{7cm}{4cm}}{\pgfpoint{7cm}{2.52cm}}
%    \pgfline{\pgfpoint{7cm}{2.52cm}}{\pgfpoint{6cm}{1.52cm}}
%    \pgfline{\pgfpoint{6cm}{1.52cm}}{\pgfpoint{5cm}{2.52cm}}
%  \end{pgfpicture}
%
%  2. Adăugăm la el element cu element...
%\end{frame}

%\begin{frame}{Studiu de caz}
%  \pgfuseimage{m0}
%
%  3. Și ajungem aici.
%\end{frame}

%\begin{frame}{Soluții}
%  \begin{itemize}[<+->]
%    \item Ne structurăm mai bine codul de la început.
%    \item Putem ajunge la buguri ce vor fi vizibile abia la integrarea a două
%    porțiuni de cod.
%    \item Sau, putem testa fiecare unitate de cod (funcție, clasă etc.)
%  \end{itemize}
%\end{frame}

%\section{Unit Testing}

%\begin{frame}{Ce presupune unit testing?}
%  \begin{description}[<+->]
%    \item[garanție]  = code that just works
%    \item[unitate] = cea mai mică bucată independentă din cod
%    \item[scop] = izolare bucăți cod, demonstrare corectitudine
%  \end{description}
%\end{frame}

%\begin{frame}{Unit testing și scrierea codului}
%  \begin{description}[<+->]
%    \item[ÎNAINTE] : detaliere aspecte proiect într-o manieră folositoare
%    \item[SCRIERE] : se evită scrierea de prea mult cod
%    \item[MODIFICARE] : garantează că nu se strică ce mergea deja
%  \end{description}
%\end{frame}

%\begin{frame}{Cum testăm?}
%  \begin{itemize}[<+->]
%    \item manual: \textbf{assert} sau \textbf{if}
%    \item automat: folosim un framework existent
%  \end{itemize}
%\end{frame}

%\begin{frame}{Exercițiu 1}
%  \begin{alertblock}{Task 1}
%    Scrieți o funcție mystrrev ce va realiza inversarea unui șir de caractere
%    dat argument. Funcția va returna un pointer către șirul rezultat.
%  \end{alertblock}
%  \pause
%  \begin{exampleblock}{Task 2}
%    Scrieți o funcție check\_mystrrev ce va testa funcția anterioară:\\
%    mystrrev(mystrrev(sir)) == sir
%  \end{exampleblock}
%  \pause
%  Ați tratat și cazul argumentelor nule?\\
%  \pause
%  Ați verificat memoria?
%\end{frame}

%\section{Test Driven Development}

%\begin{frame}{Testing vs Coding}
%  \begin{itemize}[<+->]
%    \item Pe ce punem mai mult accent? Scris teste sau scris cod?
%    \item ...
%    \item ...
%    \item Depinde
%  \end{itemize}
%\end{frame}

%\begin{frame}{Test Driven Development (TDD)}
%  \begin{itemize}[<+->]
%    \item scriem testele întâi
%    \item de fiecare dată când modificăm codul verificăm să nu crească numărul
%    testelor picate (și care sunt ele)
%    \item pică un test trecut anterior = am modificat ceva ce nu trebuia
%    modificat (\textbf{don't commit})
%    \item toate testele trecute = proiect terminat
%    \item apare un bug nou $\rightarrow$ se scriu noi teste și repetăm
%    secvența
%  \end{itemize}
%\end{frame}

%\begin{frame}{1 pic = 1000 words}
%  \pgfuseimage{m1}
%\end{frame}

%\begin{frame}{Exercițiu 2}
%  \begin{block}{Descriere}
%    Vom implementa funcția mystrcmp ce va compara dacă două șiruri de
%    caractere sunt egale. Vom începe prin a defini testul și apoi vom scrie
%    codul.
%  \end{block}
%  \begin{block}{Task 1}
%    Scrieți o funcție de test ce va verifica că mystrcmp returnează ce
%    trebuie. Nu implementați mystrcmp, scrieți doar un stub.
%  \end{block}
%  \pause
%  \begin{block}{Task 2}
%    Implementați mystrcmp, pe etape. Rulați funcția de test de fiecare dată
%    când compilați.
%  \end{block}
%\end{frame}

%\begin{frame}{Timeline}
%  \pgfuseimage{m2}
%\end{frame}

%\begin{frame}{6 motive pentru TDD}
%  \begin{enumerate}[<+->]
%    \item Focus
%    \item Știi când ai terminat
%    \item Dacă un unit merge o dată va merge mereu
%    \item Cod mult mai modular
%    \item Cod mai curat și mai ușor de înțeles
%    \item Satisfacția de a vedea că lucrurile merg
%    \item Progres vizibil
%    \item ``Documentare'' cod
%  \end{enumerate}
%\end{frame}

%\begin{frame}{No silver bullet}
%  \begin{itemize}[<+->]
%    \item Nu merge mereu
%    \item Prea multe teste
%    \item Linii de test de 100 de ori mai multe ca linii de cod utile
%    \item Teste nesemnificative
%    \item Teste pe cazuri nespecificate
%    \item Teste incorect specificate
%    \item Teste depinzând de implementare
%    \item Uneori testarea durează prea mult
%    \item Testarea manuală poate fi sărită
%    \item not thinking ahead
%  \end{itemize}
%\end{frame}

%\begin{frame}{Not thinking ahead}
%  \pgfuseimage{m3}
%\end{frame}

%\section{no TDD uses}

%\begin{frame}{Testarea codului scris deja}
%  \begin{itemize}[<+->]
%    \item goes without saying
%    \item se poate face manual, prin script sau prin cod
%    \item ignorată
%  \end{itemize}
%\end{frame}

%\begin{frame}{Integrarea cu SCM-uri}
%  \begin{itemize}[<+->]
%    \item \textbf{git bisect}
%    \item testăm manual?
%    \item putem face un script pentru testare
%  \end{itemize}
%\end{frame}

%\begin{frame}{Integrarea cu SCM-uri - Demo}
%  \begin{block}{Descriere}
%    Descărcați sursele unui proiect aflate pe git. Repository-ul are adresa:\\
%    git://gitorious.org/pali/pali.git\\
%    Avem de afișat toate palindroamele dintr-o frază.
%  \end{block}
%  \begin{block}{Problema}
%    Compilați sursele și rulați programul. Vom încerca să găsim commit-ul ce a
%    introdus eroarea folosind git bisect.
%  \end{block} \pause
%  \begin{block}{Soluție}
%    Descărcați cele două fișiere bash de la\\
%    http://swarm.cs.pub.ro/$\sim$mihai/cdl/2010/.\\
%    Rulați ut.sh și veți obține commit-ul ce a introdus greșeala. Corectați-o
%    introducând un nou commit corect (nu faceți push).
%  \end{block}
%\end{frame}

%\begin{frame}{Stress testing}
%  \begin{itemize}[<+->]
%    \item Robustețea programului
%    \item Teste peste limitele operării normale
%    \item Paralelism, sisteme de operare, middleware, biblioteci, memory leaks
%    \item Zburați cu un avion al cărui software nu a fost testat astfel?
%  \end{itemize}
%\end{frame}

%\begin{frame}{Concluzii}
%  \begin{itemize}[<+->]
%    \item Testing is good \pause \alert{if done properly}
%    \item \textit{``Testing can show the presence of bugs, but not the absence''}
%    \item suntem siguri că testăm tot?
%    \item \textit{``Testing is the process of executing a program with the intent of finding errors''}
%    \item Testarea codului duce și la cod lizibil, module separate etc.
%  \end{itemize}
%\end{frame}
%
%\begin{frame}{Întrebări?}
%\end{frame}

%\begin{frame}{Bibliografie}
%  \begin{itemize}
%    \item \href{http://www.agiledata.org/essays/tdd.html}{Test Driven Development}
%    \item \href{http://www.lispcast.com/uncategorized/6-reasons-to-develop-your-tests-first/}{Motive pentru TDD}
%    \item \href{http://stackoverflow.com/questions/301693/why-didnt-unit-testing-work-out-for-your-project}{Eșecuri TDD}
%    \item \href{http://docs.python.org/library/unittest.html}{Unit testing în Python}
%    \item \href{http://check.sourceforge.net/}{Framework pentru unit testing în C}
%    \item \href{http://www.cs.chalmers.se/~rjmh/QuickCheck/}{QuickCheck - Haskell}
%    \item \href{http://www.junit.org/}{JUnit - Java}
%    \item \href{http://www.youtube.com/watch?v=XP4o0ArkP4s}{http://www.youtube.com/watch?v=XP4o0ArkP4s}
%  \end{itemize}
%\end{frame}

\end{document}
