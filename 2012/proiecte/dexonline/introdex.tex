\documentclass[red]{beamer}
\setbeamertemplate{navigation symbols}{}
\usepackage[utf8x]{inputenc}
\usepackage{graphicx}
\usepackage{beamerthemeshadow}

\usetheme{CambridgeUS}
\setbeamercolor{itemize item}{fg=red} % all frames will have red bullets

\begin{document}
\title{DEX online - Dicționare ale limbii române}  
\author{Mihai Bărbulescu}
\date{25.02.2012} 

\begin{frame}
\titlepage
\end{frame}

\begin{frame}
\frametitle{Cuprins}
\tableofcontents
\end{frame}


\section{De ce?} 
\subsection{Importanța DEX Online}
\begin{frame}
\frametitle{De ce?}
\begin{itemize}

  \item DEX online este transpunerea pe Internet a unor dicționare de prestigiu ale limbii române.
  \vspace{0.4cm}
  \item Cu siguranță l-ați folosit și voi măcar o dată 
  \vspace{0.4cm}
  \item DEX online nu este perfect, dar a răspuns și răspunde unei nevoi stringente.
 
\end{itemize}
\end{frame}

\begin{frame}
\frametitle{Impactul social}
\begin{itemize}
 \item DEX Online e o marcă îndrăgită. Întrebați deținătorii câte felicitări primesc pe email  
 \item Veți face un mare bine culturii românești.  \pause
  \item Puțină statistică: 
	\begin{itemize}
	 \item peste {\bf 1.5 milioane} de utilizatori distincți din România folosesc sistemul
	  \item peste 11 milioane de pagini afișate lunar 
	\item Peste 10 căutări pe secundă 
	  \item $\simeq 1.500.000$ vizitatori/lună care fac $\simeq 10.000.000$ de căutări.
	\end{itemize}
\end{itemize}
\end{frame}

\section{Echipa}

\begin{frame}
\frametitle{Realizatorul DEX Online}

\begin{tabular}{l | p{10cm}}
\includegraphics[scale=0.3]{francu.jpg} & 
\begin{itemize}
   \item Cătălin Frâncu  
    \begin{itemize}
      \item Inițiatorul proiectului și programator cu experiență
     \item http://catalin.francu.com/
      \item http://catalinfrancu.blogspot.com/
     \item cata@francu.com
    \end{itemize}
   \end{itemize}
\end{tabular}
\end{frame}

\begin{frame}
\frametitle{Responsabil în cadrul CDL al proiectului}
\begin{itemize}
 \item Mihai Bărbulescu  
    \begin{itemize}
     \item 325CA
     \item b12mihai@gmail.com
      \item http://github.com/b12mihai
    \end{itemize}
\end{itemize}
\end{frame}


\section{Taskuri}
\subsection{Cunoștințe necesare}

\begin{frame}
\frametitle{Ce ar fi bine să cunoașteți}
  \begin{itemize}
   \item HTML, CSS, XML
    \item JavaScript
     \item PHP, MySQL \pause
   
  \end{itemize}

 \begin{alertblock}{Nu vă panicați!}
    Vă oferim documentația de care aveți nevoie, vă ghidăm și vă răspundem la întrebări în fiecare sâmbătă la CDL sau pe mailing list, dacă ”frige”. Oricum
înainte de a ne apuca de treabă, stabilim exact tehnologiile folosite și ce trebuie să știm.
    \end{alertblock}


\end{frame}

\subsection{La ce lucrăm?}
\begin{frame}

\frametitle{Tichete propuse}
\begin{itemize}
 \item Joculețe lingvistice - spânzurătoarea, teste de vocabular, rebus (\#257)
 \item (Mega)Test de vocabular - câte cuvinte știi? (\# 233)
   \item Crawler de texte românești (\# 243)
 \item Cele mai căutate cuvinte în ultima săptămână (\#148) \pause

 \begin{alertblock}{Already done?}
     Atunci puteți lucra la {\bf orice alt task} care nu este soluționat în prezent o dată ce reușim să rezolvăm din taskurile propuse. Se dorește să lucrăm
în echipă, la o idee și să veniți cu idei și cu sugestii!
 \end{alertblock} \pause
\begin{exampleblock}{Se caută programator voluntar}
O colaborare reușită pentru ambele părți se poate concretiza într-un contract de muncă la nivelul pieței, îndată ce DEX online va începe să angajeze.
http://dexonline.blogspot.com/2011/09/dex-online-cauta-programator.html
\end{exampleblock}


\end{itemize}
\end{frame}

\subsection{Infrastructura}
\begin{frame}
\frametitle{Cum lucrăm}
 \begin{itemize}
  \item Cod disponibil la zi (git sau svn)
  \item Toate activitățile se desfășoară pe wiki.dexonline.ro
  \item ”DEX Online Development Mailing List” $<$dexonline@lists.rosedu.org$>$
  \item Blog al proiectului: http://dexonline.blogspot.com/
 \end{itemize}

\end{frame}


\begin{frame}[plain]
\frametitle{Aveți întrebări?}
\begin{figure}
\includegraphics[scale=0.25]{question-mark.jpg} 
\end{figure}
\end{frame}

\end{document}