% vim: set tw=78 aw sw=2 sts=2 noet:
\documentclass{beamer}

%\includeonlyframes{c} % speeding up compilation speed during debug

\usepackage[utf8x]{inputenc} % diacritice
\usepackage[romanian]{babel}
\usepackage{hyperref}        % folositi \url{http://...}
\mode<presentation>
\usetheme{CDL}

\title[]{UML}
\subtitle{CDL - Cursul 7}
\institute[]{ROSEdu}
\author[]{Mihai Maruseac \\ \texttt{mihai.maruseac@rosedu.org}}

\setbeamertemplate{frametitle continuation}[from second]
\setbeamertemplate{footline}[frame number]

\pgfdeclareimage[height=6cm]{m1}{img/1}
\pgfdeclareimage[height=7cm,width=11cm]{m2}{img/2}
\pgfdeclareimage[height=5cm, width=8cm]{m3}{img/3}
\pgfdeclareimage[height=5cm]{m4}{img/4}

\begin{document}

\maketitle

\begin{frame}{Studiu de caz}
  \begin{itemize}[<+->]
    \item Cum dezvoltăm proiecte?
    \begin{itemize}[<+->]
      \item Cod, cod, cod, cod, gata
      \item Cod, cod, cod, documentație, gata
      \item cod, comentarii, cod, comentarii, documentație, gata
    \end{itemize}
    \item ``Voi când mai gândiți?''
    \item Unde proiectăm aplicația?
  \end{itemize}
\end{frame}

\begin{frame}{Modelarea Proiectelor Software}
  \begin{description}[<+->]
    \item[Modelarea:] parte esențială în orice proiect
    \item[Model:] reprezentare abstractă a sistemului
  \end{description}
  \onslide<3->De ce folosim modele?
  \begin{description}[<+->]
    \item[Înainte] verificăm dacă toate cerințele sunt acoperite
    \item[Înainte] verificăm dacă toate funcțiile sunt complete
    \item[Înainte] verificăm dacă arhitectura este robustă
    \item[După] Verificare și validare
  \end{description}
\end{frame}

\section{UML}

\begin{frame}{UML}
  \begin{itemize}[<+->]
    \item Unified Modelling Language (not User Mode Linux)
    \item Limbaj modelare, independent de procesul de dezvoltare folosit
  \end{itemize}
  \onslide<3->\begin{description}
    \item[1996] Prima versiune
    \item[1997] UML 1.0, UML 1.1
    \item[1998] UML 1.2
    \item[1999] UML 1.3
    \item[2002] UML 1.4
    \item[2004] UML 2.0
  \end{description}
\end{frame}

\begin{frame}{UML (OpenSource) tools}
  Situație ``nasoală''
  \begin{description}[<+->]
    \item[BoUML] 28.03.2010, prea buggy pentru a putea fi folosit
    \item[Astade] 11.03.2010, Tigris, wxWidgets
      \begin{itemize}
	\item generator de cod penru C, C++
	\item interfațare wxGlade, Doxygen, GIT, SVN
	\item nu se află în repo; multe deb-uri de instalat manual
	\item hint-uri de instalare (custom package source)
	\item interfață neintuitivă
      \end{itemize}
    \item[Umbrello] KDE Based, 2006, UML2
    \item[ArgoUML] 1998, nu suportă UML2
  \end{description}
  \only<12->{Scrie cineva ceva bun?}
\end{frame}

\section{Use cases}
\begin{frame}{Use case}
  \begin{itemize}[<+->]
    \item<.-> interacțiune \textit{Sistem} $\leftrightarrow$ \textit{Utilizator}
    \item<.-> interacțiune \textit{Sistem} $\leftrightarrow$ \textit{Componentă
    externă}
    \item \begin{exampleblock}{Actor}
      Rol pe care-l joacă o entitate externă (utilizatori,
      echipamente, alte sisteme, etc) în raport cu sistemul
    \end{exampleblock}
    \item \begin{exampleblock}{Scenariu}
      Secvență de pași ce descrie o interacțiune între un actor și sistem.
    \end{exampleblock}
    \item \begin{exampleblock}{Caz de utilizare (use case)}
      Abstractizare dialog actor $\leftrightarrow$ sistem, descrie
      interacțiunile fără a intra în detalii.
    \end{exampleblock}
  \end{itemize}
\end{frame}

\begin{frame}{Demo 1}
  \begin{exampleblock}{Cerință}
    Descrieți un sistem de gestiune electronică a cărților din mai multe
    biblioteci
  \end{exampleblock}
  \begin{exampleblock}{Specificații}
    \begin{itemize}
      \item 2 categorii de utilizatori: bibliotecari și abonați
      \item bibliotecarii înregistrează abonații
      \item bibliotecarii înregistrează cărți noi
      \item bibliotecarii elimină cărți din evidență
      \item abonații cer informații despre cărți
      \item abonații împrumută cărți
      \item sistemul de gestiune va folosi o interfață Web2.0
    \end{itemize}
  \end{exampleblock}
\end{frame}

\begin{frame}{Soluție 1}
  \pgfuseimage{m1}\\
  Ce lipsește?
\end{frame}

\begin{frame}{Descrierea cazurilor de utilizare}
  \begin{itemize}[<+->]
    \item Diagrama anterioară nu oferă prea multe detalii
    \item Scenariile se vor descrie separat, într-un document
  \end{itemize}
\end{frame}

\begin{frame}{Utilitate cazuri de utilizare}
  \begin{itemize}[<+->]
    \item determinarea contextului sistemului
    \item desprinderea cerințelor
    \item documentarea cerințelor
    \item proiectarea interfeței cu utilizatorii
    \item validarea arhitecturii
    \item testare
  \end{itemize}
\end{frame}

\section{Diagrame de interacțiune}

\begin{frame}{Diagrame de interacțiune}
  \begin{itemize}[<+->]
    \item Componentele proiectului nu sunt statice
    \item Evoluție în timp $\rightarrow$ diagrame de secvență
    \item Interacțiuni $\rightarrow$ diagrame de colaborare
  \end{itemize}
\end{frame}

\begin{frame}{Demo:Împrumutarea unei cărți}
\pgfuseimage{m2}
\end{frame}

\begin{frame}{Folosire}
  \begin{itemize}[<+->]
    \item definire cerințe
    \item proiectare
    \item înțelegere interacțiuni
  \end{itemize}
\end{frame}

\section{Clase și diagrame de clase}

\begin{frame}{Clasa}
  Grup de obiecte:
  \begin{itemize}[<+->]
    \item proprietăți similare (atribute)
    \item comportament comun (operații)
    \item relații comune cu alte obiecte
  \end{itemize}
\end{frame}

\begin{frame}[label=c]{O clasă}
\pgfuseimage{m3}
\end{frame}

\begin{frame}[label=c]{Relații între clase}
\pgfuseimage{m4}
\end{frame}

\begin{frame}{Exercițiu}
  \begin{exampleblock}{Cerință}
    Se cere să programați un ATM. Proiectați diagrama de clase, o diagramă a
    cazurilor de utilizare și o singură diagramă de interacțiune.
  \end{exampleblock}
\end{frame}

\section {Concluzii}

\begin{frame}{Concluzii}
  \begin{itemize}[<+->]
    \item UML este un instrument puternic pentru documentarea proiectului și
    designul acestuia
    \item Este un limbaj ușor de înțeles
    \item Ajută în discuțiile cu clientul
    \item Din păcate nu există un utilitar bun open-source (de ce?)
  \end{itemize}
\end{frame}

\begin{frame}{Întrebări?}
\end{frame}

\end{document}
