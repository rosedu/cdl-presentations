% vim: set tw=78 sts=2 sw=2 ts=8 aw et:
\documentclass{beamer}

\usepackage[utf8x]{inputenc}		% diacritice
\usepackage[romanian]{babel}
%\usepackage{color}			% highlight
%\usepackage{alltt}			% highlight
%\usepackage{code/highlight}		% highlight
\usepackage{hyperref}			% folosiți \url{http://...}
                                        % sau \href{http://...}{Nume Link}
\usepackage{verbatim}
\usepackage{tabularx}

\mode<presentation>
\usetheme{CDL}

% Titlul nu foloseşte Unicode pentru că e o problemă căreia nu i-am dat de
% cap.
\title[Wiki-uri]{Wiki-uri}
\subtitle{CDL - Cursul 5}
\institute[ROSEdu]{ROSEdu}
\date{27 martie 2010}
\author{Răzvan Deaconescu \\ \texttt{razvan@rosedu.org}}

\begin{document}

% Slide-urile cu mai multe părți sunt marcate cu textul (cont.)
\setbeamertemplate{frametitle continuation}[from second]
% Arătăm numărul frame-ului
\setbeamertemplate{footline}[frame number]

\frame{\titlepage}

\frame{\tableofcontents}


% NB: Secțiunile nu sunt marcate vizual, ci doar apar în cuprins.
\section{No\textcommabelow{t}iuni generale}

% Pentru reamintirea periodică a cuprinsului și unde ne aflăm:
\frame{\tableofcontents[currentsection]}

% Titlul unui frame se specifică fie în acolade, imediat după \begin{frame},
% fie folosind \frametitle

\begin{frame}{De ce wiki-uri?}
  \begin{itemize}
    \pause \item editare colaborativă
    \pause \item formatare facilă, link-uri facile
    \pause \item easy to install, use, edit (no training)
    \pause \item revision control
    \pause \item knowledge base (tutoriale, informații utile, documentare)
  \end{itemize}
\end{frame}

\begin{frame}{Ce este un wiki?}
  \begin{itemize}
    \pause \item site web, editare facilă
    \pause \item limbaj markup simplificat, vizualizare rapidă a informațiilor
    \pause \item wiki software (wiki engine) -- aplicație ce rulează wiki
    \pause \item suport pentru colaborare
    \pause \item istoric, revizie, autentificare, ACL
    \pause \item Ward Cunningham: ``the simplest online database that could possibly
work''
  \end{itemize}
\end{frame}

\begin{frame}{Wiki versus \ldots}
  \begin{itemize}
    \pause \item MS Office / OpenOffice
      \begin{itemize}
        \pause \item (wiki) lucru colaborativ, interfață facilă, web
        \pause \item (office) printer friendly, complex, acces controlat
    (non-professional)
      \end{itemize}
    \pause \item fișiere LaTeX într-un repository
      \begin{itemize}
        \pause \item (wiki) ușor de editat, feedback imediat, WYIWYG, web
        \pause \item (office) printer friendly, profesionist
      \end{itemize}
    \pause \item CMS
      \begin{itemize}
        \pause \item (wiki) ușor de editat, lucru colaborativ facil
        \pause \item (CMS) aspect și prezentare
      \end{itemize}
    \pause \item Google Docs
      \begin{itemize}
        \pause \item (wiki) open to public, web-friendly (link-uri etc.)
        \pause \item (gdocs) gestiune mai bună a accesului, apropiat Office
      \end{itemize}
  \end{itemize}
\end{frame}

\begin{frame}{Cazuri de utilizare}
  \begin{itemize}
    \pause \item knowledge base (comunitate, companie, proiect)
    \pause \item colaborare, editare colaborativă
    \pause \item tutoriale
    \pause \item publicare rapidă de conținut (curbă de învățare redusă)
  \end{itemize}
\end{frame}

\begin{frame}{Exemple cunoscute}
  \begin{itemize}
    \item \url{http://www.wikimatrix.org/}
    \item MediaWiki
    \item DokuWiki
    \item TWiki
    \item MoinMoin
    \item PhpWiki
    \item PmWiki
    \item integrate în alte aplicații/site-uri
  \end{itemize}
\end{frame}

\section{DokuWiki}

% Pentru reamintirea periodică a cuprinsului și unde ne aflăm:
\frame{\tableofcontents[currentsection]}

\begin{frame}{De ce DokuWiki?}
  \begin{itemize}
    \pause \item \texttt{http://www.dokuwiki.org/dokuwiki}
    \pause \item ``targeted at developer teams, workgroups and small companies''
    \pause \item comunitate, dezvoltare continuă
    \pause \item număr mare de plugin-uri
    \pause \item lightweight, file-based -- nu necesită suport de bază de date
    \pause \item ușor de instalat, configurat, personalizat
    \pause \item ``Most views'' pe WikiMatrix
(\url{http://www.wikimatrix.org/statistic/Most+Views})
    \pause \item We use it! :-P
  \end{itemize}
\end{frame}

\begin{frame}{Sintaxă Creole}
  \begin{itemize}
    \item \url{http://www.wikicreole.org/}
    \item ``common wiki markup language to be used across different wikis''
    \item suport în DokuWiki, MoinMoin, PmWiki
    \item plugin DokuWiki -- \url{http://www.dokuwiki.org/plugin:creole}
  \end{itemize}
\end{frame}

\begin{frame}{Hints}
  \begin{itemize}
    \pause \item în DokuWiki încercați folosirea sintaxei Creole acolo unde este
posibil
    \pause \item use the syntax documentation, Luke! --
\url{http://www.dokuwiki.org/syntax}
    \pause \item folosiți \texttt{<code> \ldots <\textbackslash{}code>} pentru cod,
fișiere
      \begin{itemize}
        \pause \item plasați \texttt{<code>} în continuarea ultimului cuvânt
dinainte de cod
      \end{itemize}
    \pause \item imaginile pot fi centrate, se poate folosi ``hover caption'' --
\url{http://www.dokuwiki.org/syntax\#image\_links}
    \pause \item pentru a crea o nouă pagină căutați \texttt{ns:subns:page} și apoi
creați pagina
    \pause \item pentru a șterge o pagină, ștergeți conținutul și salvați pagina
  \end{itemize}
\end{frame}

\section{Redmine Wiki}

% Pentru reamintirea periodică a cuprinsului și unde ne aflăm:
\frame{\tableofcontents[currentsection]}

\begin{frame}{De ce Redmine?}
  \begin{itemize}
    \pause \item web-based software project management system
    \pause \item repository, wiki, issues, files, documents, forum, activity,
calendar etc.
    \pause \item proiecte multiple în cadrul aceleiași instanțe
    \pause \item gestiunea utilizatorilor
  \end{itemize}
\end{frame}

\begin{frame}{Sintaxă Redmine wiki}
  \begin{itemize}
    \item \url{http://www.redmine.org/wiki/1/RedmineTextFormatting}
    \item pentru crearea unei pagini se creează un link către o pagină și apoi
se accesează
    \item există o pagină de ajutor cu sintaxa wiki-ului disponibilă în
momentul editării
  \end{itemize}
\end{frame}


\section{Concluzii}

% Pentru reamintirea periodică a cuprinsului și unde ne aflăm:
\frame{\tableofcontents[currentsection]}

\begin{frame}{Comparison chart}
  \begin{center}
    \begin{tabular}{|c|c|c|}
      \hline
      \textbf{Format} & \textbf{Creole} & \textbf{Redmine} \\
      \hline
      \hline
      heading 1 & = Nume & .h1 Nume \\
      bold & **text** & *text* \\
      italic & //text// & \_text\_ \\
      link la pagină & [[PageName$|$Nume link]] & [[PageName$|$Nume link]] \\
      URL-uri externe & [[URL$|$Nume link]] & "Nume link":URL \\
      liste neordonate & *, ** & *, ** \\
      liste ordonate & \#, \#\# & \#, \#\# \\
      nowiki & \{\{\{ \ldots \}\}\} & .bq \ldots \\
      \hline
    \end{tabular}
  \end{center}
\end{frame}

\begin{frame}{Cuvinte cheie}
  \begin{columns}
    \begin{column}[l]{0.5\textwidth}
      \begin{itemize}
        \item wiki
        \item editare colaborativă
        \item easy to install, use, customize
        \item markup language
        \item wiki engine
      \end{itemize}
    \end{column}
    \begin{column}[l]{0.5\textwidth}
      \begin{itemize}
        \item WikiMatrix
        \item DokuWiki
        \item sintaxă Creole
        \item Redmine wiki
      \end{itemize}
    \end{column}
  \end{columns}
\end{frame}

\begin{frame}{Link-uri utile}
  \begin{itemize}
    \item \url{http://www.wikimatrix.org/}
    \item \url{http://www.dokuwiki.org/syntax}
    \item \url{http://www.wikicreole.org/wiki/Creole1.0}
    \item \url{http://www.redmine.org/wiki/1/RedmineTextFormatting}
  \end{itemize}
\end{frame}

\begin{frame}{Exerciții}
  \footnotesize \url{http://swarm.cs.pub.ro/~razvan/dokuwiki/rosedu/cdl/wiki/readme}
\end{frame}


\section{\^{I}ntrebări}

% Pentru reamintirea periodică a cuprinsului și unde ne aflăm:
\frame{\tableofcontents[currentsection]}

\end{document}
