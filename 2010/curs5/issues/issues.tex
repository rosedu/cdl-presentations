% vim: set tw=78 sts=2 sw=2 ts=8 aw et:
\documentclass{beamer}

\usepackage[utf8x]{inputenc}		% diacritice
\usepackage[romanian]{babel}
%\usepackage{color}			% highlight
%\usepackage{alltt}			% highlight
%\usepackage{code/highlight}		% highlight
\usepackage{hyperref}			% folosiți \url{http://...}
                                        % sau \href{http://...}{Nume Link}
\usepackage{verbatim}
\usepackage{tabularx}

\mode<presentation>
\usetheme{CDL}

% Titlul nu foloseşte Unicode pentru că e o problemă căreia nu i-am dat de
% cap.
\title[Issue tracking]{Issue tracking}
\subtitle{CDL - Cursul 5}
\institute[ROSEdu]{ROSEdu}
\date{27 martie 2010}
\author{Răzvan Deaconescu \\ \texttt{razvan@rosedu.org}}

\begin{document}

% Slide-urile cu mai multe părți sunt marcate cu textul (cont.)
\setbeamertemplate{frametitle continuation}[from second]
% Arătăm numărul frame-ului
\setbeamertemplate{footline}[frame number]

\frame{\titlepage}

\frame{\tableofcontents}


% NB: Secțiunile nu sunt marcate vizual, ci doar apar în cuprins.
\section{Issue tracking}

% Pentru reamintirea periodică a cuprinsului și unde ne aflăm:
\frame{\tableofcontents[currentsection]}

% Titlul unui frame se specifică fie în acolade, imediat după \begin{frame},
% fie folosind \frametitle

\begin{frame}{Issues}
  \begin{itemize}
    \pause \item probleme, solicitări de rezolvare a unor probleme
    \pause \item tip problemă, autor, asignat, stare, deadline, prioritate,
descriere
    \pause \item tichete -- help desk, call center
    \pause \item issue tracking system -- gestiunea issue-urilor unei
organizații, unui proiect
      \begin{itemize}
        \pause \item aplicație software
        \pause \item interfață web, bază de date
        \pause \item asemănător cu un bug tracking system (bugtracker)
        \pause \item autentificare -- în proiectele open-source submiterea de
bug-uri e deschisă
      \end{itemize}
  \end{itemize}
\end{frame}

\begin{frame}{Issue tracking systems}
  \begin{itemize}
    \pause \item Bugzilla, MantisBT -- single purpose
    \pause \item Trac, Redmine -- multi purpose
    \pause \item SourceForge, Launchpad, Google Code, GitHub -- hosted
  \end{itemize}
\end{frame}


\section{Issue tracking în Redmine}

% Pentru reamintirea periodică a cuprinsului și unde ne aflăm:
\frame{\tableofcontents[currentsection]}

\begin{frame}{New issues}
  \begin{itemize}
    \pause \item tracker -- tipul issue-ului (configurabil)
    \pause \item subject
    \pause \item description
    \pause \item category (de configurat)
    \pause \item due date
    \pause \item priority
    \pause \item assigned to
    \pause \item attached files
  \end{itemize}
\end{frame}

\begin{frame}{View/edit issues}
  \begin{itemize}
    \pause \item list
    \pause \item summary
    \pause \item Gantt chart
    \pause \item calendar
    \pause \item filters
    \pause \item update
    \pause \item delete, move, copy (admin rights)
    \pause \item personalizare mod de afișare
  \end{itemize}
\end{frame}

\section{Concluzii}

% Pentru reamintirea periodică a cuprinsului și unde ne aflăm:
\frame{\tableofcontents[currentsection]}

\begin{frame}{Link-uri utile}
  \begin{itemize}
    \item \url{http://en.wikipedia.org/wiki/Issue\_tracking\_system}
    \item \url{http://www.redmine.org/wiki/redmine/RedmineIssues}
    \item \url{http://www.bugzilla.org/}
    \item
\url{http://en.wikipedia.org/wiki/Comparison\_of\_issue\_tracking\_systems}
    \item \url{http://www.chiark.greenend.org.uk/~sgtatham/bugs.html}
  \end{itemize}
\end{frame}

\begin{frame}{Cuvinte cheie \& întrebări}
  \begin{columns}
    \begin{column}[l]{0.5\textwidth}
      \begin{itemize}
        \item issues
        \item issue tracking
        \item bug tracking
        \item Redmine
        \item create, view, update
      \end{itemize}
    \end{column}
    \begin{column}[c]{0.5\textwidth}
      \begin{figure}
        \pause \includegraphics[scale=0.4]{img/question-mark.jpg}
      \end{figure}
    \end{column}
  \end{columns}
\end{frame}

\begin{frame}{Exerciții}
  \footnotesize \url{http://swarm.cs.pub.ro/~razvan/dokuwiki/rosedu/cdl/issues/readme}
\end{frame}


\end{document}
