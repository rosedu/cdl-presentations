% vim: set tw=78 sts=2 sw=2 ts=8 aw et:
\documentclass{beamer}

\usepackage[utf8x]{inputenc}
\PrerenderUnicode{aâîțșĂÎÂȚȘ}
\usepackage[romanian]{babel}
\usepackage{hyperref}
\usepackage{verbatim}
\usepackage{tabularx}
\usepackage{booktabs}

\mode<presentation>
\usetheme{CDL}

% Show contents at every section beginning. Ripped off from manual.
\AtBeginSection[] % Do nothing for \section*
{
  \begin{frame}<beamer>
    \frametitle{Outline}
  \tableofcontents[currentsection]
    \end{frame}
}

\renewcommand{\arraystretch}{1.3}

\title[Colaborare]{Aplicații colaborative}
\subtitle{Comunicare, Wiki, Issues}
\date{9 martie 2013}
\author[Răzvan]{Răzvan Deaconescu \\ \texttt{razvan@rosedu.org}}

\begin{document}

\setbeamertemplate{footline}[frame number]

\frame{\titlepage}

\section{Colaborare în Open Source}

\begin{frame}{De ce colaborare?}
  \begin{itemize}
    \pause
    \item sincronizare
    \item schimb de idei
    \item revizii
    \item the more, the merrier
    \item in union lies strength
  \end{itemize}
\end{frame}

\begin{frame}{De ce colaborare cu tool-uri electronice?}
  \begin{itemize}
    \pause
    \item ubiquitous
    \item acces de oriunde, oricând
    \item sincronizare independentă de poziție geografică, fus orar
    \item searchable, linkable, likeable
    \item istoric
    \item personalizare
  \end{itemize}
\end{frame}

\begin{frame}{Cum colaborăm/comunicăm?}
  \begin{itemize}
    \item real time: față în față, messager, IRC
    \item asincron: e-mail, liste de discuții, forumuri, StackOverflow
    \item colaborare: wiki, issues, bug reports
    \item repository, code reviews
    \item gist, pastebin
  \end{itemize}
\end{frame}

\begin{frame}{Liste de discuții}
  \begin{itemize}
    \item lurk before you leap
    \item folosește regulile listei
    \item vorbește la obiect
    \item less is more
    \item subiecte relevante
    \item
      \url{http://elf.cs.pub.ro/so/wiki/resurse/lista-discutii\#mailing-list-guidelines}
  \end{itemize}
\end{frame}

\section{GitHub}

\begin{frame}{Web-base Software Project Management}
  \begin{itemize}
    \item gestiunea proiectelor sofware
    \item de obicei wiki, issue/ticket tracker, roadmap, acces la repository
    \item colaborare, organizare, history
    \item client-server: Trac, Redmine
    \item hosted: SourceForge, BerliOS, Savannah, Google Code
    \item GitHub, Gitorious
  \end{itemize}
\end{frame}

\begin{frame}{De ce GitHub?}
  \begin{itemize}
    \item social coding
    \item it works!
    \item făcut de dezvoltatori pentru dezvoltatori
    \item ușor de contribuit, raportat issue-uri, creat fork-uri
    \item wiki
    \item Git
    \item issues
    \item forks, pull requests
    \item Gist
  \end{itemize}
\end{frame}

\section{Wiki-uri}

\begin{frame}{De ce wiki-uri?}
  \begin{itemize}
    \item editare colaborativă
    \item formatare facilă, link-uri facile
    \item easy to install, use, edit (no training)
    \item revision control
    \item knowledge base (tutoriale, informații utile, documentare)
  \end{itemize}
\end{frame}

\begin{frame}{Ce este un wiki?}
  \begin{itemize}
    \item site web, editare facilă
    \item limbaj markup simplificat, vizualizare rapidă a informațiilor
    \item wiki software (wiki engine) -- aplicație ce rulează wiki
    \item suport pentru colaborare
    \item istoric, revizie, autentificare, ACL
    \item Ward Cunningham: ``the simplest online database that could possibly
work''
  \end{itemize}
\end{frame}

\begin{frame}{Wiki versus \ldots}
  \begin{itemize}
    \item MS Office / OpenOffice
      \begin{itemize}
        \item (wiki) lucru colaborativ, interfață facilă, web
        \item (office) printer friendly, complex, acces controlat
    (non-professional)
      \end{itemize}
    \item fișiere LaTeX într-un repository
      \begin{itemize}
        \item (wiki) ușor de editat, feedback imediat, WYIWYG, web
        \item (office) printer friendly, profesionist
      \end{itemize}
    \item CMS
      \begin{itemize}
        \item (wiki) ușor de editat, lucru colaborativ facil
        \item (CMS) aspect și prezentare
      \end{itemize}
    \item Google Docs
      \begin{itemize}
        \item (wiki) open to public, web-friendly (link-uri etc.)
        \item (gdocs) gestiune mai bună a accesului, apropiat Office
      \end{itemize}
  \end{itemize}
\end{frame}

\begin{frame}{Cazuri de utilizare}
  \begin{itemize}
    \item knowledge base (comunitate, companie, proiect)
    \item colaborare, editare colaborativă
    \item tutoriale
    \item publicare rapidă de conținut (curbă de învățare redusă)
  \end{itemize}
\end{frame}

\begin{frame}{Exemple cunoscute}
  \begin{itemize}
    \item \url{http://www.wikimatrix.org/}
    \item MediaWiki
    \item DokuWiki
    \item TWiki
    \item MoinMoin
    \item PhpWiki
    \item PmWiki
    \item integrate în alte aplicații/site-uri (GitHub)
  \end{itemize}
\end{frame}

\begin{frame}{Sintaxă GitHub wiki}
  \begin{itemize}
    \item wiki-ul GitHub poate folosi mai multe forme de sintaxă
    \item cea implicită, folosită și pe issue-urile Git este GitHub Flavored
      Markdown
      \begin{itemize}
        \item \url{https://help.github.com/articles/github-flavored-markdown}
      \end{itemize}
    \item lightweight -- gets sh*t done
    \item menținută, în spate, în Git
  \end{itemize}
\end{frame}

\begin{frame}{Sintaxă Creole}
  \begin{itemize}
    \item \url{http://wikicreole.org/}
    \item se dorește un standard de wiki
    \item multe wiki-uri au un plugin de suport
    \item multe elemente de sintaxă împrumutate de la MediaWiki
  \end{itemize}
\end{frame}

\begin{frame}{Comparison chart}
  \begin{center}
    \begin{tabular}{@{}lcc@{}}
      \toprule
      \textbf{Format} & \textbf{Creole} & \textbf{GitHub Flavored Markdown} \\
      \midrule
      heading 1 & = Nume & \# Nume \\
      bold & **text** & **text** \\
      italic & //text// & \_text\_ \\
      link la pagină & [[PageName$|$Nume link]] & [Nume link](URL pagină) \\
      URL-uri externe & [[URL$|$Nume link]] &  [Nume link](PageName) \\
      liste neordonate & *, ** & *, $<$blank$>$ \ldots * \\
      liste ordonate & \#, \#\# & 1., $<$blank$>$ \ldots 1. \\
      nowiki & \{\{\{ \ldots \}\}\} & $```$ \ldots $```$ \\
      \bottomrule
    \end{tabular}
  \end{center}
\end{frame}

\section{Issues}

\begin{frame}{Issues}
  \begin{itemize}
    \item probleme, solicitări de rezolvare a unor probleme
    \item tip problemă, autor, asignat, stare, deadline, prioritate,
descriere
    \item tichete -- help desk, call center
    \item issue tracking system -- gestiunea issue-urilor unei
organizații, unui proiect
      \begin{itemize}
        \item aplicație software
        \item interfață web, bază de date
        \item asemănător cu un bug tracking system (bugtracker)
        \item autentificare -- în proiectele open-source submiterea de
bug-uri e deschisă
        \item în GitHub poți submite issue-uri dacă ești autetificat
      \end{itemize}
  \end{itemize}
\end{frame}

\begin{frame}{Issue tracking systems}
  \begin{itemize}
    \item Bugzilla, MantisBT -- single purpose
    \item Trac, Redmine -- multi purpose
    \item SourceForge, Launchpad, Google Code, GitHub -- hosted
  \end{itemize}
\end{frame}

\begin{frame}{GitHub -- New issues}
  \begin{itemize}
    \item title
    \item assigned to
    \item labels
    \item message
    \item milestones
    \item GitHub Flavored Markdown
    \item preview
  \end{itemize}
\end{frame}

\begin{frame}{GitHub -- View/edit issues}
  \begin{itemize}
    \item by assignee
    \item by label
    \item open/closed issues
    \item update label/assignee/milestone
    \item close/open
    \item write update -- GitHub Flavored Markdown
  \end{itemize}
\end{frame}

\section{Concluzii}

\begin{frame}{Cuvinte cheie}
  \begin{columns}
    \begin{column}[l]{0.5\textwidth}
      \begin{itemize}
        \item colaborare
        \item comunicare
        \item comunitate
        \item GitHub
        \item wiki
        \item editare colaborativă
        \item easy to install, use, customize
        \item markup language
        \item wiki engine
      \end{itemize}
    \end{column}
    \begin{column}[l]{0.5\textwidth}
      \begin{itemize}
        \item WikiMatrix
        \item GitHub wiki
        \item DokuWiki
        \item sintaxă Creole
        \item issues
        \item issue tracking
        \item bug tracking
        \item create, view, update
      \end{itemize}
    \end{column}
  \end{columns}
\end{frame}

\begin{frame}{Link-uri utile}
  \begin{itemize}
    {\footnotesize
    \item \url{https://github.com/}
    \item \url{https://gist.github.com/}
    \item \url{https://help.github.com/articles/github-flavored-markdown}
    \item \url{http://daringfireball.net/projects/markdown/syntax}
    \item \url{http://www.wikimatrix.org/}
    \item \url{http://www.dokuwiki.org/syntax}
    \item \url{http://www.wikicreole.org/wiki/Creole1.0}
    \item \url{http://en.wikipedia.org/wiki/Issue\_tracking\_system}
    \item \url{http://www.bugzilla.org/}
    \item
\url{http://en.wikipedia.org/wiki/Comparison\_of\_issue\_tracking\_systems}
    \item \url{http://www.chiark.greenend.org.uk/$\sim$sgtatham/bugs.html}
    \item
      \url{http://elf.cs.pub.ro/so/wiki/resurse/lista-discutii\#mailing-list-guidelines}
    }
  \end{itemize}
\end{frame}

\section{Întrebări}

\begin{frame}{Lucrativ}
  \begin{itemize}
    \pause \item creează-ți, dacă nu ai, cont pe GitHub
    \pause \item formați echipe de 3 persoane
    \pause \item un membru creează un proiect și îi invită pe ceilalți
      \begin{itemize}
        \item proiectul este o idee de proiect software a echipei
      \end{itemize}
    \pause \item descrieți pe wiki proiectul
      \begin{itemize}
        \item folosiți cât mai multe forme de sintaxă wiki
        \item creați cel puțin 2 pagini de wiki
      \end{itemize}
    \pause \item creați și link-ați pe wiki cel puțin un snippet Gist
    \pause \item parcurgeți descrierea echipei din dreapta voastră și creați-le
      issue-uri cu observații
    \pause \item rezolvați issue-urile cu observații
  \end{itemize}
\end{frame}

\end{document}
