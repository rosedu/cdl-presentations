% vim: set tw=78 aw:
\documentclass{beamer}

\usepackage[utf8x]{inputenc}      % diacritice
\usepackage[romanian]{babel}
\usepackage{color}                % highlight
\usepackage{alltt}                % highlight
\usepackage{code/highlight}       % highlight
\usepackage{hyperref}             % folositi \url{http://...}
                                  % sau \href{http://...}{Nume Link}
\mode<presentation>
{ \usetheme{Rochester} }          % TODO: settle this

% Titlul nu foloseşte Unicode pentru că e o problemă 
% căreia nu i-am dat de cap.
\title[Python]{Python}
\subtitle{CDL - Cursul 4}
\institute{ROSEdu}
\author{Lucian Adrian Grijincu \\ 
  \texttt{lucian.grijincu@rosedu.org}}

\begin{document}

% Slide-urile cu mai multe părţi sunt marcate cu textul (cont.)
\setbeamertemplate{frametitle continuation}[from second]
% Arătăm numărul frame-ului
\setbeamertemplate{footline}[frame number]

\frame{\titlepage}

\frame{\tableofcontents}

% NB: Secţiunile nu sunt marcate vizual, ci doar apar în cuprins.
\section{Inca un limbaj?}

% Pentru reamintirea periodică a cuprinsului şi unde ne aflăm:
\frame{\tableofcontents[currentsection]}


\begin{frame}{Încă unul?}
  \begin{itemize}
  \item ASM
  \item FORTRAN %(\textbf{încă} folosit în HPC)
  \item Pascal
  \item C
  \item C++
  \item Java \& C\#
  \item JavaScript
  \item ???
  \end{itemize}
\end{frame}


\begin{frame}{Dacă am scrie un limbaj nou am vrea ...}
  \begin{itemize}[<+->]
  \item să fie ușor de învățat
  \item compilator portabil (să rulezeze pe cât mai multe platforme)
  \item programele scrise în el să fie portabile
  \item scalabil (bun și pentru proiecte mici și mari)
  \item stabil, matur, simplu și elegant
  \item bibliotecă standard generoasă
  \item multe biblioteci terțe
  \item să fie orientat obiect (că era la modă acum câțiva ani)
  \item să fie funcțional (că e la modă acum)
  \item dezvoltare rapidă
  \item să fie emebedable (să poată fi inclus în alte programe
    pentru scripting – de ex. în jocuri)
  \item self-documenting (vedem mai târziu ce înseamnă)
  \end{itemize}
\end{frame}


\section{Interpretor interactiv}
\frame{\tableofcontents[currentsection]}


\begin{frame}{}
  \begin{itemize}
  \item permite evaluarea expresiilor fără compilare \\
    \texttt{
      >>> 1 + 1 \\ 
      1         \\
      >>> x = 1 \\ 
      >>> y = 2 \\ 
      >>> x+y   \\
      3
    }
  \item referințele nu au tip\\
    \texttt{
      >>> x = 1        \\
      >>> x            \\
      1                \\
      >>> x = 'asdf'   \\
      >>> x            \\
      'asdf'
    }
  \end{itemize}
\end{frame}

\begin{frame}{help}
  \begin{itemize}
  \item informații de ajutor pentru fiecare obiect
  \item executați \textbf{help(1)} – veți obține help-ul 
    clasei \textbf{int} asdf
  \item \textbf{\_\_add\_\_, \_\_and\_\_, \_\_div\_\_, \_\_sub\_\_} 
    sunt nume pentru \\
    \textbf{+, \&\&, /, -}
  \item executați \textbf{help(str)}
  \end{itemize}
\end{frame}



\section{Sintaxa}
\frame{\tableofcontents[currentsection]}

\begin{frame}{if - elif - else}
  \begin{itemize}[<+->]
  \item ce observați în codul următor? \\
    \input{code/if}
  \item lipsesc \{\} și ().
  \item blocurile sunt separate prin identare
  \item simplu, curat, arată a pseudocod
  \end{itemize}
\end{frame}

\begin{frame}{funcții}
  \begin{itemize}
  \item funcțiile sunt documentate prin \textit{docstrings} \\
   \small \input{code/fun}
  \item \textbf{help(factorial)}                            \\
    \texttt{
      Help on function f in module \_\_main\_\_\:           \\
      factorial(n)                                          \\
      Computes the factorial of n.                          \\
      Returns n!.
    }
  \end{itemize}
\end{frame}

\begin{frame}{clase}
  \begin{itemize}
  \item clase sunt documentate și ele prin \textit{docstrings} \\
    \small \input{code/class_light}
  \end{itemize}
\end{frame}

\begin{frame}{clase}
  \begin{itemize}
  \item metodele unei clase primesc ca prim parametru pe \textbf{self}
  \item \textbf{var}.method(param1, param2, param3)
  \item def method(\textbf{self}, param1, param2, param3)
  \item def class(\textbf{clasaParinte1}, \textbf{clasaParinte2}, etc.)
   \end{itemize}
\end{frame}


\begin{frame}{module}
  \begin{columns}[t]
    \begin{column}{4cm}
     \scriptsize { \input{code/md_impl} }
      \end{column}
    \begin{column}{4cm}
      \scriptsize { \input{code/md_user} }
    \end{column}
  \end{columns}
\end{frame}

\section{Duck typing}
\frame{\tableofcontents[currentsection]}


\begin{frame}{mac mac}
  \begin{itemize}
  \item \texttt{If it walks like a duck and quacks like a duck, I would call it a duck.}
  \item \texttt{In other words, don't check whether it \textbf{IS-a} duck: check whether it \textbf{QUACKS-like-a} duck, \textbf{WALKS-like-a} duck, etc, etc, depending on exactly what subset of duck-like behaviour you need to play your language-games with.}
  \end{itemize}
\end{frame}

\begin{frame}{interfaces}
  \begin{itemize}
  \item în alte limbaje se verifică dacă un obiect e moștenitor al unei clase
  \item în python se verifică dacă obiectul are metodele care sunt cerute
  \end{itemize}
\end{frame}

\begin{frame}{ducks}
  \tiny \input{code/duck}
\end{frame}

\section{Tipuri de date}
\frame{\tableofcontents[currentsection]}



\begin{frame}{Dicționare}
  \begin{itemize}
  \item echivalentul unui HashTable în Java
  \item inserția/ștergerea/găsirea unei valori în \textbf{O(log(n))}
  \item \small \input{code/dict}
  \end{itemize}
\end{frame}



\begin{frame}{Tupluri}
  \begin{itemize}
  \item ați vrut vreodată să întoarceți două valori dintr-o funcție?
  \item  \small \input{code/tuple}
  \item tuplurile sunt imutabile!
  \item tupluri de 0 elem?
  \end{itemize}
\end{frame}


\begin{frame}{Liste}
  \begin{itemize}
  \item \{ \} pentru dicționare, ( ) pentru tupluri, au mai rămas [ ] :)
  \item \small \input{code/lists}
  \item cu \textbf{len} se află dimensiunea listei
  \end{itemize}
\end{frame}

\begin{frame}{Liste (cont.)}
  \begin{itemize}
  \item \small \input{code/lists2}
  \end{itemize}
\end{frame}


\begin{frame}{Main()}
  \begin{itemize}
  \item C, C++, Java, etc. au \textbf{main()}
  \item \small \input{code/main}
  \item linia care începe cu \textbf{\#!} spune care este programul care execută fișierul ăsta
  \item doar când fișierul e executat \textbf{\_\_name\_\_} devine \textbf{''\_\_main\_\_''}
  \item dacă vrem ca ceva să fie executat și la încărcarea fișierului 
    (de ex. când facem \textbf{import}) punem apelul în corpul fișierului neidentat
  \end{itemize}
\end{frame}


\begin{frame}{String}
  \begin{itemize}
  \item \input{code/string1}
  \item \input{code/string2}
  \end{itemize}
\end{frame}

\begin{frame}{Exerciții}
  \begin{itemize}
  \item intrați pe git.rosedu.org și luați .tar.gz-ul ultimului commit
  \item Deschideți \texttt{curs4/python/exerciții/1\_medii.py}
  \item calculați media elementelor listei
  \end{itemize}
\end{frame}

\begin{frame}{Exerciții (cont.)}
  \begin{itemize}
  \item Deschideți \texttt{curs4/python/exerciții/2\_tuple\_list.py}
  \item scrieți trei funcții:
    \begin {itemize}
    \item Pentru fiecare tip de acțiune calculați prețul de cumpărare al portofoluilui de acțiuni
    \item Pentru fiecare tip de acțiune calculați prețul curent al portofoluilui de acțiuni
    \item Pentru fiecare tip de acțiune calculați câștigul obținut.
    \end{itemize}
  \end{itemize}
\end{frame}


\section{IO}
\frame{\tableofcontents[currentsection]}

\begin{frame}{Citire din fișiere}
  \begin{itemize}
  \item \small \input{code/file_io}
  \end{itemize}
\end{frame}

\begin{frame}{Citire din fișiere}
  \begin{itemize}
  \item \small \input{code/tabel_io}
  \end{itemize}
\end{frame}

\begin{frame}{Exerciții}
  \begin{itemize}
  \item În \texttt{curs4/python/exerciții/3\_date.in}
  \item calculați numărul de apariții al fiecărui cuvânt din text
  \item scrieți cuvintele într-un alt fișier în ordine inversă
  \end{itemize}
\end{frame}


\section{Parametri din linia de comandă}
\frame{\tableofcontents[currentsection]}

\begin{frame}{sys.arg}
  \begin{itemize}
  \item rulăm cu \textbf{./file.py param1 param2 param3}
  \item în modulul \textbf{sys} există o listă \textbf{argv}
  \item \small \input{code/param_cmdline}
  \end{itemize}
\end{frame}




\section{?}
\frame{\tableofcontents[currentsection]}

\begin{frame}{Resurse}
  \begin{itemize}
  \item help(XXX) - documentație internă
  \item Dive Into Python - carte gratuită
  \item python.org
  \end{itemize}
\end{frame}

\begin{frame}{Î \& A}
  ?
\end{frame}

\end{document}
