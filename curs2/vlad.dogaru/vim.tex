% vim: set tw=78 aw sw=2 sts=2 noet:
\documentclass{beamer}

\usepackage[utf8x]{inputenc} % diacritice
\usepackage[romanian]{babel}
\usepackage{color}			 % highlight
\usepackage{alltt}			 % highlight
\usepackage{code/highlight}	 % highlight
\usepackage{ulem} % not in the template; used for strikethrough (sout)
\usepackage{hyperref}        % folositi \url{http://...}
                             % sau \href{http://...}{Nume Link}
\mode<presentation>
{ \usetheme{Rochester} }		% TODO: settle this

% Titlul nu foloseşte Unicode pentru că e o problemă căreia nu i-am dat de
% cap.
\title{Editorul Vim}
\subtitle{CDL - Cursul 2}
\institute{ROSEdu}
\author{Vlad Dogaru \\ \texttt{ddvlad@rosedu.org}}

\begin{document}

% Slide-urile cu mai multe părţi sunt marcate cu textul (cont.)
\setbeamertemplate{frametitle continuation}[from second]
% Arătăm numărul frame-ului
\setbeamertemplate{footline}[frame number]

\frame{\titlepage}

\begin{frame}{Where do we stand?}
Până acum, la CDL, am învăţat:
\begin{itemize}
  \item<2-> ce este Free/Libre/Open/Open Source Software
  \item<3-> cum să comunicăm în cadrul unui proiect
  \item<4-> cum să automatizăm procesul de compilare
  \item<5-> cum să ţinem sursele programului undeva sigur
\end{itemize}
\onslide<6>{Când învăţăm să \textit{scriem} cod?}
\end{frame}

\begin{frame}
  \frametitle{}
  \begin{center}
  {\Huge \bfseries Acum!}
  \vspace{2cm}

  \pause De fapt, azi, dar puţin mai târziu.

  \pause Acum învăţăm cum să \textit{edităm}, nu cum să scriem cod.
  \end{center}
\end{frame}

\begin{frame}{Importanţa eficienţei}
De ce e important să scriem cod eficient:
\begin{itemize}
  \item<2-> când scriem repede, scriem mai mult
    \begin{itemize}
    \item<3-> mai mult nu înseamnă mai bine
    \item<4-> but we're sometimes ``in the zone''
    \end{itemize}
  \item<5-> dacă scriem mult şi ineficient, o să ajungem să urâm ce facem
    \begin{itemize}
    \item<6-> Da, o să ne doară mâinile. \onslide<7->{\textbf{Tare.}}
    \end{itemize}
  \item<8-> e un exerciţiu al minţii să edităm eficient
\end{itemize}
\end{frame}

\section{Introducere}
\frame{\tableofcontents}

\begin{frame}{De ce Vim?}
\begin{itemize}
  \item<1-> \sout{un editor text-mode este mult mai eficient decât un IDE
  grafic}
  \item<2-> \sout{vim face treaba mai bine decât emacs}
  \item<3-> contează cu ce ne obişnuim
    \begin{itemize}
    \item<4-> dar ar trebui să ne obişnuim cu ceva flexibil şi portabil
    \end{itemize}
\end{itemize}
\end{frame}

% Pentru reamintirea periodică a cuprinsului şi unde ne aflăm:
%\frame{\tableofcontents[currentsection]}

\end{document}
