\documentclass{beamer}

\usepackage[utf8x]{inputenc}
\usepackage[romanian]{babel}
\usepackage{color}
\usepackage{alltt}
\usepackage{code/highlight}
\mode<presentation>
\usetheme{Rochester}

\title[Makefile]{Makefile}
\institute{ROSEdu}
\author{Andrei Soare}

\begin{document}

\setbeamertemplate{frametitle continuation}[from second]
\setbeamertemplate{footline}[framenumber]

\frame{\titlepage}

\frame{\tableofcontents}

\section{Ce este Makefile?}
    \frame{\tableofcontents[currentsection]}
    
    \begin{frame}{Definitie}
    \begin{itemize}
    \setlength{\itemsep}{0.8cm}
    \item Este un fisier care se include in orice proiect pentru a automatiza procesul de compilare al surselor.\\
    \item Compilarea efectiva: comanda make, care citeste un set de reguli din Makefile
    \end{itemize}
    \end{frame}

    \begin{frame}{De ce Makefile ?}
    \begin{itemize}
    \setlength{\itemsep}{0.4cm}
    \item Scuteste de compilarea manuala a fiecarei surse.
    \item Daca modificam surse si vrem sa recompilam proiectul, make va recompila doar fisierele modificate
    \item Asigura o anumita ordine a compilarii surselor - rezolva dependintele
    \item Instalarea foarte rapida si usoara a programelor din surse.\\Exemplu: Kernel-ul de Linux, - instalare usoara, in cativa pasi:\\\vspace{0.2cm}\noindent
\ttfamily
\hlstd{}\hlline{\ \ \ \ 1\ }\hlslc{\#!/bin/bash}\\
\hlline{\ \ \ \ 2\ }\hlstd{\\
\hlline{\ \ \ \ 3\ }MAX}\hlsym{=}\hlstd{}\hlnum{10}\\
\hlline{\ \ \ \ 4\ }\hlstd{}\\
\hlline{\ \ \ \ 5\ }\hlkwa{for\ }\hlstd{i\ }\hlkwa{in\ }\hlstd{\$}\hlsym{(}\hlstd{}\hlkwc{seq\ }\hlstd{}\hlnum{1\ }\hlstd{}\hlkwd{\$\{MAX\}}\hlstd{}\hlsym{);\ }\hlstd{}\hlkwa{do}\\
\hlline{\ \ \ \ 6\ }\hlstd{}\hlstd{\ \ \ \ \ \ \ \ }\hlstd{}\hlkwb{echo\ }\hlstd{}\hlstr{"Testing\ \$\{i\}"}\hlstd{}\\
\hlline{\ \ \ \ 7\ }\hlkwa{done}\\
\hlline{\ \ \ \ 8\ }\hlstd{}\\
\hlline{\ \ \ \ 9\ }\hlkwb{exit\ }\hlstd{}\hlnum{0}\hlstd{}\\
\mbox{}
\normalfont

    \end{itemize}
    \end{frame}

\section{Procesul de instalare al pachetelor}
    \frame{\tableofcontents[currentsection]}

    \begin{frame}{Comparatie - Sisteme de Operare}
    \begin{itemize}
    \setlength{\itemsep}{0.5cm}
    \item Pe Windows: double click, next, next, ... next, finish
    \item Pe MacOS: double click, drag \& drop :)
    \item Pe Linux: \ttfamily./configure \&\& make \&\& make install\normalfont
    \end{itemize}
    \end{frame}

    \begin{frame}{Pe Linux}
    \begin{itemize}
    \setlength{\itemsep}{0.7cm}
    \item {\ttfamily tar -x[zj]f nume\_arhiva}
    \item {\ttfamily ./configure}\\$\rightarrow$ verifica detalii despre sistem, dependinte, etc.\\$\rightarrow$ creeaza un Makefile pentru compilarea proiectului
    \item {\ttfamily make}\\$\rightarrow$ compileaza fiecare sursa din proiect $\Rightarrow$ executabile
    \item {\ttfamily make install}\\$\rightarrow$ copiaza in sistem executabilele si alte fisiere necesare rularii programului (ex: in {\ttfamily /usr/bin})
    \end{itemize}
    \end{frame}

    \begin{frame}{Practice}
    \par Luati codul sursa al programului hello si instalati-l.
    \vspace{0.6cm}
    \par $\rightarrow$ {\ttfamily apt-get source hello}
    \end{frame}

\section{Crearea unui Makefile}
    \frame{\tableofcontents[currentsection]}

    \begin{frame}{Structura}
    \begin{itemize}
    \setlength{\itemsep}{0.6cm}
    \item comentarii
    \item variabile / macrouri
    \item reguli
    \end{itemize}
    \end{frame}

    \begin{frame}[allowframebreaks]
    \frametitle{Variabile}
    \begin{itemize}
    \setlength{\itemsep}{0.6cm}
    \item Declarare: {\ttfamily nume\_variabila = valoare1 valoare2 ...}
    \item Folosire: expandarea unei variabile se face astfel: {\ttfamily\$(nume\_varabila)}
    \begin{item}
    Exemplu:\\{\ttfamily CFLAGS = \$(include\_dirs) -O\\include\_dirs = -Ifoo -Ibar}
    \end{item}
    \begin{item}
    Variabile implicite:\\
    \$@ se expandeaza la numele target-ului\\
    \$\^{} se expandeaza la lista de cerinte\\
    \$$<$ se expandeaza la prima cerinta
    \end{item}
    \begin{item}
    Exista 2 moduri de a declara variabile:
        \begin{itemize}
        \item \ttfamily\ NUME = valori\normalfont\\Atribuirea valorii se face de fiecare data cand se expandeaza variabila
        \item \ttfamily\ NUME := valori\normalfont\\Variabilei ii va fi atribuita o valoare o singura data, la definitie
        \end{itemize}
    \end{item}
    \item Functii utile pentru declararea variabilelor:\\\begin{itemize}\item wildcard\item patsubst\end{itemize}
    \item Variabilele pot fi definite/suprascrise si la rularea comenzii make. Exemplu:\\{\ttfamily make "CFLAGS = -O2 -g -Wall"}
    \end{itemize}
    \end{frame}

    \begin{frame}[allowframebreaks]
    \frametitle{Reguli}
    \begin{itemize}
    \setlength{\itemsep}{0.3cm}
    \item sintaxa unei reguli:\\ \ttfamily target: dependinte\\ <tab>comenzi\\ <tab>comenzi\\ ...\normalfont
    \item target - numele fisierului care se va obtine in urma comenzilor\\target-uri tipice: install, clean
    \item dependinte - fisiere necesare comenzilor care urmeaza
    \begin{item} Problema: target-ul ``\texttt{clean}'' care sterge toate executabilele: daca exista deja un fisier cu numele clean ?\\Solutia: .PHONY\\
    \vspace{0.3cm}
    \noindent
\ttfamily
\footnotesize
\hlstd{}\hlline{\ \ \ \ 1\ }\hlslc{\#!/usr/bin/env\ python}\\
\hlline{\ \ \ \ 2\ }\hlstd{}\hlkwa{import\ }\hlstd{pygtk\\
\hlline{\ \ \ \ 3\ }pygtk}\hlsym{.}\hlstd{}\hlkwd{require}\hlstd{}\hlsym{(}\hlstd{}\hlstr{'2.0'}\hlstd{}\hlsym{)}\\
\hlline{\ \ \ \ 4\ }\hlstd{}\hlkwa{import\ }\hlstd{gtk}\\
\hlline{\ \ \ \ 5\ }\hlkwa{class\ }\hlstd{Hello}\hlsym{:}\\
\hlline{\ \ \ \ 6\ }\hlstd{}\hlstd{\ \ \ \ }\hlstd{}\hlkwa{def\ }\hlstd{}\hlkwd{del\textunderscore event\textunderscore f}\hlstd{}\hlsym{(}\hlstd{self}\hlsym{,\ }\hlstd{widget}\hlsym{,\ }\hlstd{event}\hlsym{,\ }\hlstd{data\ }\hlsym{=\ }\hlstd{}\hlkwa{None}\hlstd{}\hlsym{):}\\
\hlline{\ \ \ \ 7\ }\hlstd{}\hlstd{\ \ \ \ \ \ \ \ }\hlstd{}\hlkwa{return\ False}\\
\hlline{\ \ \ \ 8\ }\hlstd{}\hlstd{\ \ \ \ }\hlstd{}\hlkwa{def\ }\hlstd{}\hlkwd{destroy\textunderscore sgn\textunderscore f}\hlstd{}\hlsym{(}\hlstd{self}\hlsym{,\ }\hlstd{widget}\hlsym{,\ }\hlstd{data\ }\hlsym{=\ }\hlstd{}\hlkwa{None}\hlstd{}\hlsym{):}\\
\hlline{\ \ \ \ 9\ }\hlstd{}\hlstd{\ \ \ \ \ \ \ \ }\hlstd{gtk}\hlsym{.}\hlstd{}\hlkwd{main\textunderscore quit}\hlstd{}\hlsym{()}\\
\hlline{\ \ \ 10\ }\hlstd{}\hlstd{\ \ \ \ }\hlstd{}\hlkwa{def\ }\hlstd{}\hlkwd{\textunderscore \textunderscore init\textunderscore \textunderscore }\hlstd{}\hlsym{(}\hlstd{self}\hlsym{):}\\
\hlline{\ \ \ 11\ }\hlstd{}\hlstd{\ \ \ \ \ \ \ \ }\hlstd{self}\hlsym{.}\hlstd{w\ }\hlsym{=\ }\hlstd{gtk}\hlsym{.}\hlstd{}\hlkwd{Window}\hlstd{}\hlsym{(}\hlstd{gtk}\hlsym{.}\hlstd{WINDOW\textunderscore TOPLEVEL}\hlsym{)}\\
\hlline{\ \ \ 12\ }\hlstd{}\hlstd{\ \ \ \ \ \ \ \ }\hlstd{self}\hlsym{.}\hlstd{w}\hlsym{.}\hlstd{}\hlkwd{show}\hlstd{}\hlsym{()}\\
\hlline{\ \ \ 13\ }\hlstd{}\hlstd{\ \ \ \ \ \ \ \ }\hlstd{self}\hlsym{.}\hlstd{w}\hlsym{.}\hlstd{}\hlkwd{connect}\hlstd{}\hlsym{(}\hlstd{}\hlstr{"delete\textunderscore event"}\hlstd{}\hlsym{,\ }\hlstd{self}\hlsym{.}\hlstd{del\textunderscore event\textunderscore f}\hlsym{)}\\
\hlline{\ \ \ 14\ }\hlstd{}\hlstd{\ \ \ \ \ \ \ \ }\hlstd{self}\hlsym{.}\hlstd{w}\hlsym{.}\hlstd{}\hlkwd{connect}\hlstd{}\hlsym{(}\hlstd{}\hlstr{"destroy"}\hlstd{}\hlsym{,\ }\hlstd{self}\hlsym{.}\hlstd{destroy\textunderscore sgn\textunderscore f}\hlsym{)}\\
\hlline{\ \ \ 15\ }\hlstd{}\hlstd{\ \ \ \ }\hlstd{}\hlkwa{def\ }\hlstd{}\hlkwd{main}\hlstd{}\hlsym{(}\hlstd{self}\hlsym{):}\\
\hlline{\ \ \ 16\ }\hlstd{}\hlstd{\ \ \ \ \ \ \ \ }\hlstd{gtk}\hlsym{.}\hlstd{}\hlkwd{main}\hlstd{}\hlsym{()}\\
\hlline{\ \ \ 17\ }\hlstd{hello\ }\hlsym{=\ }\hlstd{}\hlkwd{Hello}\hlstd{}\hlsym{()}\\
\hlline{\ \ \ 18\ }\hlstd{hello}\hlsym{.}\hlstd{}\hlkwd{main}\hlstd{}\hlsym{()}\hlstd{}\\
\mbox{}
\normalfont
\normalsize

    \end{item}
    \end{itemize}
    \end{frame}

    \begin{frame}{Reguli Implicite}
    \begin{itemize}
    \setlength{\itemsep}{0.5cm}
    \item Make are un set de reguli implicite. Acestea pot fi vizualizate cu \ttfamily make -p\normalfont
    \item Comanda care trebuie rulata poate fi detectata implicit. De exemplu, pentru regula \ttfamily hello.o: hello.c\normalfont \ se va considera implicit comanda:\\ {\ttfamily \$(CC) \$(CFLAGS) -c -o \$@ \$$<$}
    \end{itemize}
    \end{frame}

    \begin{frame}{Exemplu}
    \input{code/makefile}
    \end{frame}

    \begin{frame}[allowframebreaks]
    \frametitle{Ce face make?}
    \begin{itemize}
    \setlength{\itemsep}{0.6cm}
    \item Inainte de a citi fisierul Makefile, utilitarul make isi incarca setul de reguli si variabile implicite.
    \item OBS: se poate executa make si fara a exista un Makefile, el incarcandu-si doar regulile implicite. Exemplu:\\ \vspace{0.4cm}\input{code/06}
    \item In momentul rularii comenzii make i se poate preciza target-ul care se doreste a fi obtinut (ex: make install). Daca acesta nu este precizat, este considerat implicit primul target intalnit în fisierul Makefile folosit.
    \item La executie, make compara timpul ultimei schimbari al target-ului cu timpii dependintelor si daca oricare din dependinte este mai noua, target-ul va fi refacut.
    \end{itemize}
    \end{frame}

    \begin{frame}{Practice}
    \begin{itemize}
    \item In folderul ... aveti un mic proiect. Faceti un Makefile care sa contina reguli pentru compilarea surselor, pentru install si pentru clean
    \end{itemize}
    \end{frame}

\section{Alternative}
    \frame{\tableofcontents[currentsection]}

\section{Further Reading}
    \frame{\tableofcontents[currentsection]}

    \begin{frame}{Further Reading}
    \begin{itemize}
    %must make the smaller
    \item http://www.gnu.org/software/make/manual/make.html
    \item http://cs.pub.ro/~so/index.php?section=Laboratoare\&file=01.\%20Introducere\#GNU\_Make
    \item http://www.hsrl.rutgers.edu/ug/make\_help.html
    \item http://www.opussoftware.com/tutorial/TutMakefile.htm
    \end{itemize}
    \end{frame}

\end{document}
