% vim: set tw=78 aw:
\documentclass{beamer}

\usepackage[utf8x]{inputenc} % diacritice
\usepackage[romanian]{babel}
\usepackage{color}			 % highlight
\usepackage{alltt}			 % highlight
\usepackage{code/highlight}	 % highlight
\usepackage{hyperref}        % folositi \url{http://...}
% sau \href{http://...}{Nume Link}
\mode<presentation>
{ \usetheme{Rochester} }		% TODO: settle this

% Titlul nu foloseşte Unicode pentru că e o problemă căreia nu i-am dat de
% cap.
\title[Debian packaging]{Debian packaging}
\subtitle{CDL - Cursul 5}
\institute{ROSEdu}
\author{Lucian Adrian Grijincu \\ \texttt{lucian.grijincu@rosedu.org}}

\begin{document}

% Slide-urile cu mai multe părţi sunt marcate cu textul (cont.)
\setbeamertemplate{frametitle continuation}[from second]
% Arătăm numărul frame-ului
\setbeamertemplate{footline}[frame number]

\frame{\titlepage}

\frame{\tableofcontents}

\begin{frame}{... să nu așteptăm mai târziu}
  \begin{beamerboxesrounded}[lower=block body,shadow=true]{}
    \large {\texttt{\textit{sudo apt-get install} devscripts wget cdbs \\
      build-essential fakeroot patchutils debhelper}}
  \end{beamerboxesrounded}
\end{frame}


% NB: Secţiunile nu sunt marcate vizual, ci doar apar în cuprins.
\section{Notiuni introductive}

% Pentru reamintirea periodică a cuprinsului şi unde ne aflăm:
\frame{\tableofcontents[currentsection]}


% Titlul unui frame se specifică fie în acolade, imediat după \begin{frame},
% fie folosind \frametitle
\begin{frame}{Utilitate uzuale}
  \begin{itemize}[<+->]
  \item \textbf{./configure} - script generat cu autotools
    \begin{itemize}
    \item verifică dependențele programului
    \item folosit pentru a trece peste diferențele între sisteme tip UNIX
    \item generază un \textit{Makefile} dintr-un \textit{Makefile.ac} (un template)
    \end{itemize}
  \item \textbf{diff+patch} - \pause{amintirile mă chinuiesc ...}
  \item \textbf{make} - construiește executabilul final din surse
  \item \textbf{???} - generare pachet .deb
  \item \textbf{dpkg} - manager de pachete pentru Debian
  \item \textbf{apt-*}, \textbf{aptitude} - interfețe de nivel înalt pentru dpkg
  \end{itemize}
\end{frame}

\begin{frame}{dpkg - manager de pachete pentru Debian}
  \begin{itemize}
  \item pentru toate pachetele de care știe ține minte starea pachetului: 
    instalat complet, doar fișierele de configurație rămase, instalare începută,
    dar neterminată, etc.
  \item \textbf{-i/\texttt{--}install f.deb} - instalare fișier .deb
  \item \textbf{-r/\texttt{--}remove pachet} - dezinstalare \textit{pachet}
  \item \textbf{-P/\texttt{--}purge pachet} - dezinstalare \textit{pachet} și a fișierelor de configurație
  \item \textbf{-L/\texttt{--}listfile pachet} - listarea fișierelor instalate de \textit{pachet}
  \item \textbf{-c/\texttt{--}contents f.deb} - listarea fișierelor dintr-un .deb
  \item \textbf{-S/\texttt{--}search file} - caută pachetele care instalează fișierul \textit{file}
  \begin{columns}[t]
    \begin{column}{4cm}
      \begin{beamerboxesrounded}[lower=block body,shadow=true,width=3cm]{\$ dpkg -S /bin/ls}
        coreutils: /bin/ls
      \end{beamerboxesrounded}
      \end{column}
    \begin{column}{4cm}
      \begin{beamerboxesrounded}[lower=block body,shadow=true,width=5cm]{\$ dpkg -S /usr/bin/dos2unix }
        tofrodos: /usr/bin/dos2unix
      \end{beamerboxesrounded}
    \end{column}
  \end{columns}
  \end{itemize}    
 \end{frame}


\begin{frame}{apt}
  \begin{itemize}
  \item \textbf{apt-cache dump} - listează toate pachetele din cache
  \item \textbf{apt-cache policy} - listează repo-urile din care e un pachet, versiunea instalată, alte versiuni disponibile
    \begin{beamerboxesrounded}[lower=block body,shadow=true,width=8cm]{}
      apt-cache policy firefox \\
      \texttt{   firefox:}\\
      Instalat: 3.0.7+nobinonly-0ubuntu0.8.10.1 \\
      Candidează: 3.0.8+nobinonly-0ubuntu0.8.10.2
    \end{beamerboxesrounded}
  \item \textbf{apt-cache show} - listează informații despre un pachetul binar (maintainer, dependențe, descriere, etc.)
  \item \textbf{apt-cache showsrc} - listează informații despre un pachetul sursă (maintainer, dependențe, descriere, etc.)
  \item \textbf{apt-cache rdepends} - listează toate pachetele care depind de acesta
  \end{itemize}
\end{frame}


\section{Împachetarea unui proiect nou}
\frame{\tableofcontents[currentsection]}

\begin{frame}{hello - un proiect simplu care afișează ``Hello World!''}
  \begin{itemize}
    \begin{beamerboxesrounded}[lower=block body,shadow=true]{1\_wget.sh - modificați numele și adresa înainte de rulare}
      \small \input{code/1_wget}
    \end{beamerboxesrounded}
  \item 4 - luăm sursele proiectului
  \item 7 - numele e important: \textbf{packagename\_version}.\textit{orig.tar.gz}
  \item 11 - dh\_make va crea un director \textit{debian/} și o serie 
    de scripturi predefinite și ne va întreba detalii despre tipul proiectului
  \end{itemize}
\end{frame}

\begin{frame}{dh\_make}
    \begin{beamerboxesrounded}[lower=block body,shadow=true]{dh\_make pune întrebări despre proiect și maintainer}
      \small \input{code/out_1_dh_make}
    \end{beamerboxesrounded}
\end{frame}

\begin{frame}{debian\/ scripts}
  \begin{itemize}
    \begin{beamerboxesrounded}[lower=block body,shadow=true]{2\_rm\_unneeded.sh - delete unneeded files from debian\/}
      \small \input{code/2_rm_unneeded}
    \end{beamerboxesrounded}
  \item \textbf{README.Debiam} - README specific pachetului de debian (nu e README-ul proiectului)
  \item \textbf{dirs} - folosit de \textit{dh\_installdirs} pentru a crea directoare
  \item \textbf{info} - folosit de \textit{dh\_installinfo} pentru a crea un fisier \texttt{info}
  \item \textbf{docs} - folosit de \textit{dh\_installdocs} pentru a instala documentația programului
  \item \textbf{*.ex *.EX} - extensii folosite în cazuri speciale
  \end{itemize}
\end{frame}


\begin{frame}{debian/changelog}
%  \begin{beamerboxesrounded}[lower=block body,shadow=true]{changelog\_template - șablon}
%    \scriptsize \input{code/changelog_template}
%  \end{beamerboxesrounded}
  \begin{beamerboxesrounded}[lower=block body,shadow=true]{}
    \scriptsize \input{code/changelog_hello}
  \end{beamerboxesrounded}
  \begin{itemize}
  \item \textbf{unstable} e denumirea versiunii de Debian pentru care se crează pachetul. 
    Poate fi înlocuit cu \textit{jaunty} dacă vrem să facem pachet petru Ubuntu Jaunty.
  \item \textbf{urgency} - cât de important e să se facă actualizarea pachetului 
    (low, medium, high, emergency, critical)
  \item cu două spații și * sunt sumarizate modificările (liniile următoare, cu identare mai
    mare descriu pe larg modificările)
  \item numele este \textbf{2.1.1-1}. \textbf{-1} a fost adăugat pentru a marca faptul că 
    acesta este primul pachet .deb creat pentru versiunea 2.1.1
  \end{itemize}
  
\end{frame}

\begin{frame}{debian/control}
  \begin{beamerboxesrounded}[lower=block body,shadow=true]{}
    \small \input{code/control_hello}
  \end{beamerboxesrounded}
\end{frame}

\begin{frame}{generare .deb}
  \begin{itemize}
  \begin{beamerboxesrounded}[lower=block body,shadow=true]{}
    debuild
  \end{beamerboxesrounded}
  \item va rezulta un fișier \texttt{hello\_2.1.1-1\_i386.deb}
  \item fișierul va fi semnat cu cheia gpg corespunzătoare numelui și 
    adresei de email din \texttt{debian/changelog}
  \end{itemize}
\end{frame}

\begin{frame}{Î \& R}
  \begin{centering}
    \Huge ?    \par
  \end{centering}
\end{frame}


\end{document}
